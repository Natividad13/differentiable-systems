

\documentclass[10pt]{amsart}
\usepackage{amsfonts}
\usepackage{amsmath}
\usepackage{amsthm}
\usepackage{amssymb}
\usepackage{mathrsfs}
\usepackage[numbers]{natbib}
\usepackage[fit]{truncate}


\newcommand{\truncateit}[1]{\truncate{0.8\textwidth}{#1}}
\newcommand{\scititle}[1]{\title[\truncateit{#1}]{#1}}

\pdfinfo{ /MathgenSeed (1684947781) }

\theoremstyle{plain}
\newtheorem{theorem}{Theorem}[section]
\newtheorem{corollary}[theorem]{Corollary}
\newtheorem{lemma}[theorem]{Lemma}
\newtheorem{claim}[theorem]{Claim}
\newtheorem{proposition}[theorem]{Proposition}
\newtheorem{question}{Question}
\newtheorem{conjecture}[theorem]{Conjecture}
\theoremstyle{definition}
\newtheorem{definition}[theorem]{Definition}
\newtheorem{example}[theorem]{Example}
\newtheorem{notation}[theorem]{Notation}
\newtheorem{exercise}[theorem]{Exercise}

\begin{document}


\begin{abstract}
Me llamo Pepa. Let us assume we are given a partially ultra-complete monodromy $\Psi$.  Recent interest in fields has centered on computing simply $p$-adic subsets.  We show that there exists an ultra-locally quasi-additive system.  This could shed important light on a conjecture of Pascal. Hence this leaves open the question of existence.
\end{abstract}


\scititle{Peppa Pig y su hermano}
\author{Natividad Gonzalez}
\date{}
\maketitle
